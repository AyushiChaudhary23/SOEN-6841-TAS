\documentclass[a4paper,12pt]{report}

\usepackage{geometry}
\usepackage{titlesec}
\usepackage{enumitem}
\usepackage{graphicx}

\geometry{
    left=1in,
    right=1in,
    top=1in,
    bottom=1in,
}

\titleformat{\chapter}[display]
  {\normalfont\bfseries}{}{0pt}{\Huge}

\setlist[enumerate]{itemsep=0mm}

\begin{document}

\title{Report on Dealing with Uncertainty}
\author{Ayushi Chaudahry}
\date{\today}

\maketitle

\tableofcontents

\chapter{Abstract}

This extensive research digs into the difficult area of decision-making under uncertainty, with a special emphasis on its importance in the field of software engineering. The article explores the issues that software engineers encounter when faced with insufficient or contradictory information and provides a systematic framework to help them make successful decisions in such situations. The framework's goal is to give software engineering students and professionals with a practical guide to navigating uncertainty, consequently increasing decision-making skills and contributing to the success of software projects.
% Your abstract content

\chapter{Introduction}
\section{Motivation}

The inspiration for researching the subject of coping with uncertainty in software engineering stems from the daily experiences of software engineering students and professionals. Uncertainty is an ever-present factor in the dynamic and continuously growing world of software development. Software developers frequently face scenarios in which judgements must be made in the absence of comprehensive or consistent facts. These decisions frequently have far-reaching consequences, influencing the quality, timeliness, and success of software initiatives. Understanding how to cope with uncertainty successfully is not just desirable; it is required for the growth and stability of the software engineering sector.
\\
Uncertainty is particularly difficult in software engineering because software systems are intangible, sophisticated, and prone to continuous change. It's an area where precision and accuracy are critical, and minor mistakes or miscalculations can have serious effects, such as project delays, budget overruns, and unsatisfied stakeholders.

% Your motivation content
\section{Problem Statement}
The inquiry is focused on the difficult task of making well-informed and successful judgements in the face of inadequate, inconsistent, or constantly changing information. The capacity to make accurate judgments under ambiguity is critical in the context of software engineering, where precision and accuracy are crucial.
\\
Software engineers are confronted with various forms of uncertainty, including:
\begin{itemize}
    \item Requirements Uncertainty: Many software projects start with incomplete or ambiguous specifications. This uncertainty can result in misunderstandings and rework, affecting project deadlines and costs.
    \item Technical Uncertainty: As technology evolves, software engineers face new challenges in selecting technologies, tools, and platforms. Decisions in this area can have long-term consequences.
    \item Uncertainty in Market and User: Software systems are designed to serve specific markets and users. Predicting user needs and market trends is inherently difficult.
    \item Uncertainty in Resources and Schedules: Managing resources, timelines, and budgets can be unpredictable as a result of unforeseen events or changes in project scope.

\end{itemize}
% Your problem statement content
\section{Objectives}
\begin{itemize}
    \item \textbf{Comprehensive Framework}: Provides a comprehensive framework for dealing with uncertainty in software engineering, covering a wide range of uncertainties from design requirements to market dynamics.
    \item \textbf{Improved Decision Making}: Equipping Software Engineering Students, Professionals, and Managers with the Information, Tools, and Strategies They Need to Improve Their Decision Making.
    \item \textbf{Effective Project Management}: Promoting Effective Project Management Practices in Software Engineering by Addressing the Enormous Challenges of Uncertainty.
\end{itemize}
\chapter{Background Material}
\section{Uncertainty in Software Engineering}
% Your content on uncertainty in software engineering
\section{The Role of Software Engineers}
% Your content on the role of software engineers
\section{The Student Perspective}
% Your content on the student perspective

\chapter{Methods \& Methodology}
\section{Approaching the Problem}
\subsection{Start by Defining the Problems and Goals}
% Your content on defining problems and goals
\subsection{Collect Information from Multiple Sources}
% Your content on collecting information from multiple sources
\subsection{Outline the Options}
% Your content on outlining options
\subsection{Be Cognizant of the Risks}
% Your content on being cognizant of the risks
\section{Techniques for Analysis}
% Your content on techniques for analysis

\chapter{Results Obtained}
\section{Under What Conditions}
% Your content on conditions where the framework is effective
\section{Constraints}
% Your content on constraints
\section{Quality Assessment}
% Your content on quality assessment

\chapter{Conclusions and Future Work}
\section{Suggested Improvements}
% Your content on suggested improvements
\section{Limitations to Solution}
% Your content on limitations
\section{The Agile Perspective}
% Your content on the Agile perspective

\end{document}
