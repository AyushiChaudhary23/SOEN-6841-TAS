\documentclass[a4paper,12pt]{report}

\usepackage{geometry}
\usepackage{titlesec}
\usepackage{enumitem}
\usepackage{graphicx}

\geometry{
    left=1in,
    right=1in,
    top=1in,
    bottom=1in,
}

\titleformat{\chapter}[display]
  {\normalfont\bfseries}{}{0pt}{\Huge}

\setlist[enumerate]{itemsep=0mm}

\begin{document}

\title{Report on Dealing with Uncertainty}
\author{Ayushi Chaudahry}
\date{\today}

\maketitle

\tableofcontents

\chapter{Abstract}

This extensive research digs into the difficult area of decision-making under uncertainty, with a special emphasis on its importance in the field of software engineering. The article explores the issues that software engineers encounter when faced with insufficient or contradictory information and provides a systematic framework to help them make successful decisions in such situations. The framework's goal is to give software engineering students and professionals with a practical guide to navigating uncertainty, consequently increasing decision-making skills and contributing to the success of software projects.
% Your abstract content

\chapter{Introduction}
\section{Motivation}

The inspiration for researching the subject of coping with uncertainty in software engineering stems from the daily experiences of software engineering students and professionals. Uncertainty is an ever-present factor in the dynamic and continuously growing world of software development. Software developers frequently face scenarios in which judgements must be made in the absence of comprehensive or consistent facts. These decisions frequently have far-reaching consequences, influencing the quality, timeliness, and success of software initiatives. Understanding how to cope with uncertainty successfully is not just desirable; it is required for the growth and stability of the software engineering sector.
\\
Uncertainty is particularly difficult in software engineering because software systems are intangible, sophisticated, and prone to continuous change. It's an area where precision and accuracy are critical, and minor mistakes or miscalculations can have serious effects, such as project delays, budget overruns, and unsatisfied stakeholders.

% Your motivation content
\section{Problem Statement}
The central issue under investigation is the difficult task of making well-informed, effective decisions in the unpredictable world of software engineering. Field decision-makers are frequently confronted with incomplete, inconsistent, or rapidly changing information. This presents a formidable challenge, particularly in a context where precision is critical and errors can have serious consequences. As a result, the ability to make decisions in the face of uncertainty is critical.
\\ Uncertainty in software engineering manifests itself in a variety of ways, including elusive project requirements, an ever-changing technological landscape, volatile market dynamics, and unpredictability of resources and schedules. The software engineering community deals with these uncertainties on a daily basis, and the goal of this report is to investigate, analyze, and synthesize strategies for navigating this difficult terrain.
\newpage


% Your problem statement content
\section{Objectives}
The report outlined herein is driven by the following set of objectives:
\begin{itemize}
    \item \textbf{Comprehensive Framework}: Provides a comprehensive framework for dealing with uncertainty in software engineering, covering a wide range of uncertainties from design requirements to market dynamics.
    \item \textbf{Improved Decision Making}: Equipping Software Engineering Students, Professionals, and Managers with the Information, Tools, and Strategies They Need to Improve Their Decision Making.
    \item \textbf{Effective Project Management}: Promoting Effective Project Management Practices in Software Engineering by Addressing the Enormous Challenges of Uncertainty.
\end{itemize}
The findings of this study are intended to benefit not only software engineering professionals, but also the broader software engineering community, by providing practical guidance and insights into effective decision-making under uncertainty. We hope to achieve these goals by facilitating software project success, stimulating innovation, and strengthening the resilience of software engineering professionals in the face of ever-changing challenges.
% Your objectives content

% \chapter{Background Material}
% \section{Uncertainty in Software Engineering}
% % Your content on uncertainty in software engineering
% \section{The Role of Software Engineers}
% % Your content on the role of software engineers
% \section{The Student Perspective}
% % Your content on the student perspective

\chapter{Methods \& Methodology}
We outline the systematic approach used to address the problem of decision-making under uncertainty. The methodologies and techniques used are intended to provide effective strategies for navigating complex decision-making scenarios.

\begin{figure}[h!]
  \includegraphics[scale=1.0]{general_dm_process.png}
  \caption{General process for Decision making \cite{lunenburg2010decision}}
  % \label{fig:boat1}
\end{figure}

\section{Approaching the Problem}
To address uncertain decision-making, we've devised a structured framework with these key steps:
\subsection{Start by Defining the Problems and Goals}

Kepner and Tregoe developed a problem analysis method, emphasizing the importance of the first stage of decision-making—identifying the problem. According to their framework, the precision with which the problem is defined has a significant impact on the decision's quality\cite{lunenburg2010decision}. The method of problem analysis includes: 
\\(1) problem identification,\\(2) definition of what the problem is and is not,\\ (3) prioritizing the problem, and \\(4) testing for cause-effect relationships.
\\
\\\textit{The Importance of Well-Defined Problems:} In software engineering, poorly defined problems frequently lead to project failure. It is critical to devote time to defining problems, comprehending stakeholder needs, identifying potential risks, and aligning problems with project objectives. This step helps to avoid "bikeshedding," which occurs when discussions focus on implementation details before agreeing on the need for a solution.



% Your content on defining problems and goals
\subsection{Collect Information from Multiple Sources}
% Your content on collecting information from multiple sources
Effective software engineering decision-making requires gathering insights and perspectives from a variety of stakeholders, including developers, designers, product owners, and end users. Collaborative input aids in the formation of a comprehensive view of the problem.

\textit{Collaboration Among Stakeholders:} Collaboration is essential in software engineering. Learning to gather information from a variety of sources, such as team members and clients, prepares students for future professional careers.

\subsection{Outline the Options}
% Your content on outlining options
There are frequently multiple approaches to solving a problem in software engineering. This step entails developing and testing multiple solutions or approaches to ensure that software engineering students and professionals consider a wide range of options.

\textit{Accepting Creativity:} Software engineering is a creative field. Successful software engineers are able to think outside the box and consider a variety of options.

\subsection{Be Cognizant of the Risks}
% Your content on being cognizant of the risks
Risk assessment is a critical component of software project management. In the context of software development, identifying potential risks, their consequences, and their likelihood is critical. This step encourages a thorough assessment of the risks associated with each decision.

\textit{Risk Management as a Skill:} Learning to assess and manage risks is a skill that an individual will use throughout his software engineering career. Software Engineer can gain practical experience by identifying potential project risks, assessing the severity of those risks, and developing mitigation strategies.

\section{Techniques for Analysis}
% Your content on techniques for analysis

\chapter{Results Obtained}
\section{Under What Conditions}
% Your content on conditions where the framework is effective
\section{Constraints}
% Your content on constraints
\section{Quality Assessment}
% Your content on quality assessment

\chapter{Conclusions and Future Work}
\section{Suggested Improvements}
% Your content on suggested improvements
\section{Limitations to Solution}
% Your content on limitations
\section{The Agile Perspective}
% Your content on the Agile perspective

\chapter{References}
\begin{thebibliography}{9}

\bibitem{vroom1973leadership}
Vroom, V. H., \& Yetton, P. W. (1973). \textit{Leadership and decision-making.} University of Pittsburgh Press.

\bibitem{lunenburg2010decision}
Lunenburg, F. C. (2010). \textit{The decision-making process}. National Forum of Educational Administration \& Supervision Journal, 27(4).


% Add more citations as needed

\end{thebibliography}
\end{document}
